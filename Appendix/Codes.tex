\chapter{Codes}
You may have done some coding in your thesis. You can share it with the verbatim package, but it looks a lot nicer to have a specific environment for code blocks. You can even include language specific syntax highlighting. The style file mypythonhighlight.sty in the working directory is set up for python, but you can find packages for other languages too!. I made a custom code environment which gives you a caption over the code block and lists it automatically in the List of Codes. 

The simplest way to include code is to type it directly in the .tex file in the \verb=\begin{python}= environment defined by the .sty file. Just like with figures, tables, and equations, you can label and reference them. Just like with the verbatim environment, indentation is preserved, as displayed by the (useless) example, Code \ref{code:hello}

\begin{code}\caption{Hello!} \begin{python}
    print("Hello World") #comment
    try:
        a=2/x
    except ZeroDivisionError:
        print('undefined')
\end{python}\label{code:hello}\end{code}

You might be discussing a single aspect of your code in the body of your thesis. Inline codes like \pyth{import numpy} are very useful for doing this, distinguishing the commands from regular text by using the code font and syntax coloring.

My preferred way of including codes in the document is by using the custom \verb=\inputpython= command. Code \ref{code:fstrings} displays lines 1 to 3 of test.py, located in the py subdirectory. It's an example of a very nice python feature, fstrings. 

\begin{code}\caption{F strings}
\inputpython{py/test.py}{1}{3}
\label{code:fstrings}\end{code}

You can customize the syntax highlighting by digging into mypythonhighlight.sty. This may be nice if you are using a specific library and want the keywords in that library to be highlighted as keywords, but the .sty file doesn't identify them as keywords by default.