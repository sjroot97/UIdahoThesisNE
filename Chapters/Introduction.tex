\chapter{Introduction}
\label{Chapter:Introduction}
\LaTeX \; offers a lot of very nice features, including acronyms and citations. Dr. Trefor Bazett on Youtube has a great \href{https://www.youtube.com/playlist?list=PLHXZ9OQGMqxcWWkx2DMnQmj5os2X5ZR73}{playlist} to learn a lot of features.

\section{Verbatim}
Some symbols and words in \LaTeX \; are interpreted as commands. In some cases you can escape the commands, but for others, the verbatim package can be handy. \verb=\command= would normally be interpreted as a command by the compiler, but since I wrapped it with the inline verbatim environment it is interpreted literally. You can also do multiline verbatim sections, which may be useful for including code:

\begin{verbatim}
    \command
    \othercommand
\end{verbatim}

The package also supports multiline comments, which allows block of \LaTeX  code to remain in your .tex file without being generated into your document:

\begin{comment}
    This is my comment.
    Note that it can span multiple lines.
    This is very useful.
    \end{comment}

\section{Citations}
You can easily manage your citations using BibTex. Lamarsh is a good textbook for nuclear engineers \cite{Lamarsh}. The information is entered in the file References.bib in the working directory, and is called using the command \verb=\cite{Lamarsh}= It automatically populates in the References section in the format specified either by a built in commmand, or a custom format defined by a .bst file in the working directory, in this case nsf.bst. This is handy because it automatically reorders the bibliography either alphabetically or in order of in text reference.

\section{Custom Commands}
The file uidahomastersthesis.cls contains some custom commands that will be useful. You can add your own, but I've included some common ones. \verb=\flinak= prints \flinak. Custom commands can include arguments, which may be optional. \verb=\UF= prints \UF, while \verb=\UF[6]= prints \UF[6]. \verb=\U= prints \U, while \verb=\U[235]= prints \U[235], and \verb=\U[238]= prints \U[238]. \verb=\Xe= prints \Xe, while \verb=\Xe[136]= prints \Xe[136] and \verb=\Xe[]= prints \Xe[]

\section{Acronyms}
In general, acronyms should be fully spelled out for at least the first usage. The commands \verb=\acs=, \verb=\acl=, \verb=\acf= are defined in the .cls file, and print the short version, long version, and both respectively. Doing this instead of typing out the acronym is good, because it automatically populates the List of Acronyms, and links the acronym to the List of Acronyms. It takes a little setup, defining acronyms in the .cls file, but it is worth it in the long run.

You call the full acronym \acf{lwr} with \verb=\acf{lwr}=. Later on, you can call the short version (\acs{lwr}) with \verb=\acs{lwr}= or the long version (\acl{lwr}). with \verb=\acl{lwr}=.