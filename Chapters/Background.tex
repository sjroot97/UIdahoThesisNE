\chapter{Background}
\label{Chapter:Background}

You're probably gonna need to do some math. You can do inline equations like $a^2+b^2=c^2$ by wrapping with dollar signs. Especially useful for inline equations is \verb=\nicefrac=, which is a package loaded, and does diagonal fractions instead of vertical. This can keep inline equations looking nice and uncramped. $K.E. = \nicefrac{mv^2}{2}.$ instead of $K.E. = \frac{1}{2}mv^2.$ You will also want separate equations like Equation \ref{einstein}.

\begin{equation}\label{einstein}
    e=mc^2
\end{equation}

Curly braces are used in equation \ref{exp} for multi character sub/superscripts. It also shows how to use greek characters. There are a ton of other features for \LaTeX \; equations that you can find with Google.

\begin{equation}\label{exp}
    -1 = e^{i\pi} 
\end{equation}

Referencing equation numbers using \verb=\ref{label}= is handy because it updates the number if the numbering changes \eg you add another equations.


